\documentclass[11pt,twoside]{article}
\pagestyle{empty}
%\input{/home/alberto/latex/general_en.tex}

\usepackage{amssymb}
\usepackage{graphicx}
\usepackage{makeidx}
\usepackage{amsmath}
\usepackage{amsfonts}
%\usepackage{cclicenses}
\usepackage[linktocpage]{hyperref}
\usepackage[usenames,dvipsnames]{color}
\usepackage{xmpincl}
\usepackage[latin1]{inputenc}
\usepackage[spanish]{babel}

%\setlength{\parskip}{2ex plus 0.5ex minus 0.2ex}
\addtolength{\textheight}{3cm}
\addtolength{\voffset}{-2cm}
\addtolength{\textwidth}{2cm}
\addtolength{\evensidemargin}{-2.5cm}

\newcommand{\git}{\texttt{GIT}}
\newcommand{\srv}{\texttt{git.cpt.univ-mrs.fr}}
\newcommand{\ssh}{\texttt{ssh}}
\newcommand*{\OriginalQuotation}{}
\let\OriginalQuotation\quotation
\renewcommand*{\quotation}{\OriginalQuotation\small\sf}


\begin{document}

\title{Elecciones locales y auton�micas 2011}
\author{Gente}
\maketitle

\begin{abstract}
Un peque\~no an�lisis sobre las elecciones locales y auton�micas
espa�olas del 2011, dedicando especial atenci�n a los posibles efectos
de las movilizaciones del 15m.
\end{abstract}

\tableofcontents

\section{Introduction}




\end{document}

